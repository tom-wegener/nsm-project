\chapter{Einleitung}

Für das Modul Netzwerk- und System-Management (NSM) sollen die Studierenden des Studiengangs "Informatik Master" der HTWK Leipzig die Infrastruktur einer Fakultät erstellen, die sich außerhalb des Universitäts-netzes befindet.

Das Projekt soll mit Hilfe von Netkit als virtuelle Infrastruktur realisiert werden und anschließend über Ansible konfiguriert werden. Zuletzt soll ein Bericht zu dem Netzwerk und der Umsetzung verfasst werden.

\section{Umgebung}

Das Projekt wurde nicht innerhalb der VM umgesetzt, sondern für bessere Performance direkt auf einem privaten Rechner. Auf dem Rechner ist eine Linux-Distribution, die auf Ubuntu 18.10 basiert, installiert. Dabei hat Ansible die Version 2.7.8 und nutzt die Python-Version 3.6.7. Anstatt der originalen Netkit-Version wurde \href{https://netkit-ng.github.io/}{Netkit-ng} genutzt, welches ein neueres Filesystem nutzt, das auf Debian wheezy basiert. Das war notwendig, um die notwendige Version von Python zur Verfügung zu haben, da Ansible mindestens Python der Version 2.7 benötigt.

Die Verbindung zwischen dem Host bzw. Ansible und den Netkit-VMs wird über SSH realisiert in einer Agent-less Architektur.

Durch die trotzdem veraltete Version des Netkit-File-Systems können einige Sachen nicht wie im Konzept beschrieben umgesetzt werden. Durch ein Update des File-Systems auf eine neuere Debian-Version, kann dies ermöglicht werden. Für dieses Projekt wurde versucht, die bereits vorhandene Software so wenig wie möglich zu verändern. Dementsprechend sind auch nur relevante Applikationen auf den VMs aktuell, es wurde kein generelles Update ausgeführt.
