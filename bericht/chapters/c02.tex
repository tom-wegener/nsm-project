\chapter{Konzept}
Es soll die Infrastruktur einer externen Fakultät realsiert werden.

\section{Anforderungen}
Die Fakultät ist in einem weit von der Universität entfernten Gebäudekomplex untergebracht, die Infrastruktur für die Fakultät muss dementsprechend als eigene, unabhängige Infrastruktur behandelt werden. Diese Infrastruktur soll alle notwendigen Dienste der Fakultät beinhalten, wie ein Web-Auftritt, ein Mail-Service und die AAA-Services beinhalten.
Das Netzwerk der Fakultät erhält Internet und Anbindung an die Universität über eine sogenannte Dark Fiber-Leitung, diese ist eine bisher ungenutzte und angemietet Leitung eines bereits verlegten Netzes.

Außerdem sollen die einzelnen Server über SNMP über Icinga, Zabbix oder Ganglia überwachbar sein.

\section{Konzeption}
Für das Netzwerk-Konzept wurde eine Struktur mit mehreren getrennten Subnetzen gewählt, das eine einfache Administration gewährleistet, eine Erweiterung erleichtert und es erleichtert sicherheits-relevante Änderungen schnell umzusetzen.

Die Infrastruktur besteht grundlegend aus drei Subnetzen, sowie einem großen Netzwerk. Die Verbindung der Subnetze wird über das Kern-Netzwerk bzw. den Kern-Router ermöglicht. Zusätzlich zu dem Kern-Router hat jedes Subnetzwerk einen Router und der Anschluss über die Dark-Fiber-Leitung wird auch über einen Router realisiert. Die drei Subnetzwerke sind die demilitarisierte Zone (DMZ), das Server-Netzwerk und das Client-Netzwerk.
Verbunden wird alles über das Kern-Netzwerk, welches für sich auch ein eigenes Subnetz darstellt. Über dieses Netzwerk werden die verschiedenen Netzwerke verbunden. Die gesamte Infrastruktur ist vereinfacht in der Abbildung \ref{fig:model} grafisch dargestellt.

\begin{figure}
  \includegraphics[width=\linewidth]{pictures/netzwerk-diagramm.jpg}
  \caption{Modell der Infrastruktur}
  \label{fig:model}
\end{figure}

In der Abbildung ist pro Netzwerk nur ein Router eingezeichnet, um die Übersichtlichkeit zu verbessern. Es sollte bei einer Umsetzung bedacht werden, dass es mindestens zwei gibt, um einerseits einen Ersatz bei einem Ausfall zu haben, aber auch, um  ein Load-Balancing zu ermöglichen.

Die DMZ stellt alle Dienste bereit, die über das Internet erreichbar sein sollen, also einen Web-Server, einen Mail-Server und einen AAA-Server. Jedoch werden kritische Daten nicht auf Servern in der DMZ gespeichert, sondern sind auf Servern in dem Server-Netzwerk gespeichert, die jedoch über Server aus der DMZ erreichbar sind.

Das Server-Netzwerk ist für die Bereitstellung von Daten und Diensten innerhalb der Fakultät zuständig. Zur Sicherheit sollten keine Daten aus diesem Subnetz direkt den Dark-Fiber-Router passieren, sondern maximal über einen Server der DMZ abgefragt werden und dann weitergeleitet werden.
Das Netzwerk enthält den für den AAA-Services benötigten Server, die benötigten Datenbanken, die Incinga-Instanz, sowie den Foreman-Server. Außerdem können spezifische weitere Services in dem Netz realisiert werden.

Die Rechner der Angestellten der Fakultät, sowie der Studierenden der Fakultät, sind alle über das Client-Netzwerk verbunden. Die Administration erfolgt ebenfalls über einen Rechner im Client-Netz oder über den Administrations-Rechner im Server-Netzwerk.

Generell soll das Netzwerk durch das Tool \href{www.theforeman.org}{Foreman} gemanaged werden. Foreman ermöglicht die Administration von Hosts und kann auch für die Administration von großen Netzwerken samt Routern, Clients und Servern genutzt werden. Über eine Web-Oberfläche wird es ermöglicht einen Überblick zu behalten, sowie die Provisionierung und die Konfiguration abzuwickeln.
Das Administrationstool wickelt die Konfiguration eigentlich über Puppet ab, kann jedoch durch Plugins erweitert werden, wie dem Ansible-Plugin. 

Sollte die Software kein anderes Betriebssystem als Ubuntu-Server benötigen kann zusätzlich das kommerzielle Management-Tool Landscape von Canonical benutzt werden. Es ermöglicht eine einfache Administration verschiedener Instanzen und kann die Auslastung, Zustand der Hardware und den Sicherheitszustand anzeigen. Landscape basiert aus einer Agent-Architektur, erfordern also einen Client auf den zu verwaltenden Rechnern.

Außerdem können Foreman und Landscape auch bei der Realisierung von FCAPS helfen.

\section{FCAPS}
FCAPS ist ein Modell für das Management von Netzwerken bzw. Infrastrukturen, es ist ein Akronym für Fault-Management, Configuration-Management, Administration and Accounting Management, Performance Management und Security Management. FCAPS wird in diesem Fall über verschiedene Tools und mit Hilfe von Foreman realisiert.

Es wurde sich bewusst für FCAPS entschieden und nicht für FAB oder ITIL, da FCAPS sehr ausführlich die wichtigsten Bestandteile der Netzwerk-Management behandelt.


\subsection{Fault-Management}
Fault-Management bedeutet das frühzeitige Erkennen und Beheben von Fehlern innerhalb des Netzwerkes, das kann durch z.B. eine Icinga-Instanz realisiert werden. Alternativ kann auch ein anderes Werkzeug, wie z.B. Foreman, genutzt werden, welches mehrere Teile von FCAPS in sich vereinigt.

Foreman überprüft die Anwesenheit und Funktionalität von Hosts durch puppet-nodes, außerdem kann durch ein Plugin zentralisiertes Logging umgesetzt werden. Sollte ein Fehler auftreten, der unter eine bestimmte Einstufung, wie z.B. ''kritisch'' fällt, kann eine Benachrichtigung gesendet werden, außerdem wird auch ein vereinfachtes Handeln unterstützt, solange eine Verbindung zum Host besteht, können Logs abgerufen werden, Konfigurationen erneut angewendet werden oder Updates gefahren werden.

Für ein übersichtliches Logging müssen die Logs in verschiedene Stufen eingeteilt werden. Es empfiehlt sich, diese einerseits nach Gefahr für die weitere Funktionalität einzuteilen, wie auch in die verschiedene Komponenten. So können Logs, wie ''critical: infrastructure: lost connection to mail-server'' schnell verstanden werden.

\subsection{Configuration Management}
Das Konfigurationsmanagement wird über Ansible abgewicklet. Dabei werden die einzelnen benötigten Dienste als Konfigurationen in playbooks angelegt, die einen entsprechenden Namen haben, sollte ein Host ausfallen, kann in der host-Datei ein neuer Host der entsprechenden Gruppe hinzugefügt werden. Theoretisch können die Konfigurationen der Router ebenfalls über playbooks abgespeichert werden. 

Durch eine Integration von Ansible in Foreman über das dazugehörige Plugin kann die Ausführung von playbooks einfach umgesetzt werden, sowie die Ausführung überwacht werden. Die Ergebnisse der Ausführung werden bei Foreman je Host aufgelistet.

\subsection{Administration and Accounting Management}
''Administration and Accounting Management'' beinhaltet das Management der Accounts und der für die Abrechnung benötigten Nutzungsdaten. Das beinhaltet die AAA-Services. Die Nutzenden werden dafür in verschiedene Gruppen mit dazugehörigen Zugriffsrechten eingeteilt. Zusätzlich gibt es für jede Gruppe noch eine zusätzliche Gruppe in der Nutzende eingeteilt werden, die zusätzliche Rechte erhalten. Sollte eine Studentin z.B. zusätzliche Rechte benötigen, weil sie im Fachschaftsrat ist und dort die Website, die auf Fakultätsservern liegt, betreut, erhält zusätzliche Rechte, aber wird dafür nicht in die Gruppe der Website-Administrierenden aufgenommen.


\subsection{Performance Management}
''Performance Management'' bedeutet die Überwachung der Performance aller Komponenten des Netzwerkes. Dabei muss die Auslastung der Server, die Gesundheit und das Alter von Festplatten und anderen (Netzwerk-)Komponenten überwacht werden.

Grundlegendes Performanmce-Monitoring kann über Landscape abgewickelt werden, aber auch die Icinga-Instanz kann eine grundlegende Überwachung ermöglichen. 
Der Foreman-Server kann ähnliche Daten wie die Icinga-Instanz bereitstellen.

\subsection{Security Management}
Security Management bedeutet sich über die Sicherheit der Infrastruktur bewusst zu sein und diese so gut wie möglich zu verbessern. Das beinhaltet die physische, sowie die digitale Sicherheit.

Die physische Sicherheit wird unter Anderem durch eine Zugangskontrolle zu den Räumen in denen die Server stehen erreicht. Jedoch werden noch weitere Schritte benötigt.

Die digitale Sicherheit kann unter Anderem durch IPSec realisiert werden, außerdem muss ein Überblick über wichtige Sicherheits-Updates beibehalten werden und diese durch das Konfigurations-Management aufgespielt werden.

Die Authorisierung für die Dienste und das Netzwerk wird über den AAA-Server realisiert.

Alle über das Internet erreichbaren Dienste können nur über https erreicht werden, während bestimmte Dienste, wie ein File-Server, nur über ein VPN erreichbar sind.

Außerdem wird der Zugriff auf z.B. das Server-Netzwerk aus dem Internet eingeschränkt. Die es kann nur aus dem Client-Netzwerk und der DMZ auf Daten, die auf diesen Servern liegen zugegriffen werden. Außerdem werden über die beiden Firewalls die Zugriffe zusätzlich beschränkt und verschiedenen IP-Adressen geblockt.

Über Landscape kann überwacht werden, welche Sicherheits-Updates die Software, die auf den Rechnern installiert ist, noch ausstehend sind.
