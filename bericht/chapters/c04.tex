\chapter{Zusammenfassung}
Leider ist Netkit ein älteres Projekt, welches damit auch ältere Software als Basis verwendet. Besonders lässt sich das ältere File-System und der ältere Kernel betonen. Bei beiden ist die neueste offizielle Version von 2016, diese ist jedoch nicht stabil und voll funktionsfähig.

Eine andere Version, also Netkit-NG, hat ebenfalls eine veraltete Debian-Version und auch einen alten Kernel. Durch das erhöhte Alter ist die Kompatibilität zu einigen Programmen nicht gewährleistet.

Trotzdem kann die grundlegende Infrastruktur damit virtuell errichtet und ausprobiert werden.

Die Nutzung von Subnetzwerken verbessert die Sicherheit und vereinfacht die Organisation und Administration des Netzwerkes. Die Einteilung der benötigten Teile in die vier Schwerpunkte Clients, Server, DMZ und Internet- bzw. Dark-Fiber-Anbindung passt zu diesem Ansatz.

Die Verwendung von verschiedenen Programmen zur Verwaltung des Netzwerkes verkompliziert zwar das Management, da verschiedene Software benutzt werden muss und nicht alle zentral in einem Dashboard kontrolliert werden kann. Die erwähnte Software sind nur Vorschläge, wie das Netzwerk administriert werden kann, sie müssen mit genaueren Anforderungen evaluiert werden. Foreman bietet hier jedoch einen Lösungsansatz für das erste Problem, weil sich durch Plugins viel Funktionalität in dem Programm vereint oder sich vereinen lässt.

